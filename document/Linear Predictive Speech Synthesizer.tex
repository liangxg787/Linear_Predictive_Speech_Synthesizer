\documentclass[11pt]{article}
\usepackage{amsmath,amssymb,amsthm} % AMS styles for extra equation formatting
\usepackage{graphicx} % for including graphics files
\usepackage[numbers,sort]{natbib} % for better references control
\usepackage{hyperref} % for hyperlinks within the paper and references
\usepackage{titling}  % Include the titling package
\usepackage{fontspec}  % Allows for system fonts
\usepackage[top=2cm, bottom=2cm, left=2cm, right=2cm]{geometry}  % Set margins on all sides

%%%%%%%%%%%%%%%%%%%%%%%%%%%%%%%%%%%%%%%%%%%%%%%%%%%%%%%%%%%%%%%%%%%%%%%%%%%%%%

\hypersetup{colorlinks=true, linkcolor=blue,  anchorcolor=blue,
citecolor=blue, filecolor=blue, menucolor=blue, pagecolor=blue,
urlcolor=blue}

%%%%%%%%%%%%%%%%%%%%%%%%%%%%%%%%%%%%%%%%%%%%%%%%%%%%%%%%%%%%%%%%%%%%%%%%%%%%%%

\newcommand{\todo}[1]{\vspace{5 mm}\par \noindent
\marginpar{\textsc{Todo}}
\framebox{\begin{minipage}[c]{0.90 \textwidth}
\tt \flushleft #1 \end{minipage}}\vspace{5 mm}\par}

%%%%%%%%%%%%%%%%%%%%%%%%%%%%%%%%%%%%%%%%%%%%%%%%%%%%%%%%%%%%%%%%%%%%%%%%%%%%%%

\graphicspath{{graphics/}}

\newtheorem{theorem}{Theorem}[section]
\newtheorem{proposition}[theorem]{Proposition}
\newtheorem{lemma}[theorem]{Lemma}
\newtheorem{corollary}[theorem]{Corollary}
\newtheorem{definition}[theorem]{Definition}

%\renewcommand{\qedsymbol}{$\blacksquare$} % for filled square at end of proof
%\numberwithin{equation}{section} % for the 1.1, 1.2 equation number style
%\setlength{\parindent}{0em} % don't indent paragraphs
%\setlength{\parskip}{1em} % add spacing between paragraphs
%\linespread{1.6} % double-spacing

\setmainfont{Arial}

%%%%%%%%%%%%%%%%%%%%%%%%%%%%%%%%%%%%%%%%%%%%%%%%%%%%%%%%%%%%%%%%%%%%%%%%%%%%%%

\begin{document}

\title{Linear Predictive Speech Synthesizer}
% \title{%
% Linear Predictive Speech Synthesizer \\
% {\large EEEM030 - Speech & Audio Processing & Recognition \\
% Assignment 1}}
% \author{Xiaoguang Liang \\
%   6844178}
% \date{\today}
% \maketitle
% \newpage

\begin{titlepage}
  \centering  % Center everything on the title page
  \vspace*{\fill}  % Add flexible vertical space at the top to push the title down

  {\Huge\bfseries\thetitle}  % Set the title in large, bold font
  \vskip 1em  % Add some space between title and author

  {\Large\itshape EEEM030 - Speech \& Audio Processing \& Recognition \\
  Assignment 1}  % Set the author name in a slightly smaller font
  \vskip 1em  % Add space between author and date
  
  {\normalsize\slshape Xiaoguang Liang}  % Set the author name in a slightly smaller font
  \vskip 0.5em  % Add space between author and date

  {\normalsize\slshape 6844178}  % Set the author name in a slightly smaller font
  \vskip 1em  % Add space between author and date
  
  {\normalsize\slshape \today}  % Set the date in a smaller font
  
  \vspace*{\fill}  % Add flexible vertical space at the bottom to center the content
\end{titlepage}

% Ensure all figures/tables are processed before the next page (Contents page)
% \clearpage
% Suppress any floats (figures, tables) from appearing on the next page
\suppressfloats

\tableofcontents

\begin{abstract}
This is the abstract.
\end{abstract}

%%%%%%%%%%%%%%%%%%%%%%%%%%%%%%%%%%%%%%%%%%%%%%%%%%%%%%%%%%%%%%%%%%%%%%%%%%%%%%

\section{Introduction}

We have a table of contents displayed now, to help with writing the
paper, but it is normally removed before submitting to a journal.

References can be cited either directly in the text with
\verb+\citet+, to refer the reader to a paper like that of
\citet{Giles2003}, or they can be cited parenthetically
\citep{Adams2002} with \verb+\citep+. If DOI numbers are in the
references and \verb+pdflatex+ is used then the resulting PDF will
have clickable links to the papers in the References.

%%%%%%%%%%%%%%%%%%%%%%%%%%%%%%%%%%%%%%%%%%%%%%%%%%%%%%%%%%%%%%%%%%%%%%%%%%%%%%
\section{First section}

To include a figure we use a floating environment which allows
\LaTeX\ to put the figure anywhere it wants, and then we refer to it
by number as figure \ref{fig:sample}.

\begin{figure}
\begin{center}
\includegraphics[width=8cm]{"1_10_s.bmp.png"}
\end{center}
\caption{\label{fig:sample} This is the figure caption. Figure
  captions should be long and descriptive because people actually read
  them in papers, unlike most of the text.}
\end{figure}

If \verb+pdflatex+ is used then figures can be PDF, JPEG, PNG, or
other formats, but cannot be EPS. If plain \verb+latex+ is used, then
figures can only be EPS. For this reason it's typically best to
include both EPS and non-EPS versions of each figure. Under Linux the
commands \verb+epstopdf+ and \verb+convert+ are helpful with this.

%%%%%%%%%%%%%%%%%%%%%%%%%%%%%%%%%%%%%%%%%%%%%%%%%%%%%%%%%%%%%%%%%%%%%%%%%%%%%%

\subsection{First subsection}

Equations should use either the \verb+equation+ environment for single
line equations:
\begin{equation}
\label{eqn:stokes_thm}
\int_{\Omega} \mathbf{d}\alpha = \int_{\partial \Omega} \alpha
\end{equation}
or one of the \verb+amsmath+ environments (\verb+align+,
\verb+multline+, etc) for more complex equations. The \verb+align+
environment is particularly good:
\begin{align}
\label{eqn:newton}
m\vec{a} &= \sum_{i = 1}^N F_i(\vec{x}) \\
\label{eqn:newton_expanded}
&= F_1(\vec{x}) + F_2(\vec{x}) + \cdots + F_N(\vec{x})
\end{align}

\begin{theorem}
\label{thm:big_result}
This is how we state a theorem. Lemmas and corollaries are similar.
\end{theorem}

\begin{proof}
And this is the proof of our amazing result.
\end{proof}

\todo{MW: This is a ToDo box. Comments in here are not part of the
  paper, but are a discussion about the paper. Start ToDo boxes with
  your initials so that everyone can keep track of who said what. Once
  the issue in a ToDo box has been resolved then anyone should delete
  it, not just the person who wrote it.}

\todo{MW: To reply to a ToDo box, just put your own after it.}

\paragraph{Paragraph title.}
\verb+\paragraph+s can be used to break up long blocks of text and
help the reader. Paragraph titles should have the first word
capitalized and then the others lowercase, and they should end with a
period.

\paragraph{Formatting units.}
Following Knuth, units should be written like $v =
10\,\text{m}/\text{s}$ or $p =
20\,\text{kg}\,\text{m}\,\text{s}^{-1}$. The second form is generally
preferred.

\paragraph{Acronyms.}
We should try to avoid acronyms wherever possible. The only acronyms
that should be used are those that are so common that they are more
easily recognized than their definitions, such as SO(3) or TCP/IP. As
Richard Murray is fond of saying, defining an acronym just to save
typing in a paper is very bad for a reader. We can read a familiar
sequence of words almost as fast as we can read an acronym, and much
faster if we have to pause to remember what the acronym is actually
for. Generally speaking, we should limit ourselves to at most one or
two new acronyms within a paper, and they should be reserved for
things like algorithm names that we hope other people will also use.

%%%%%%%%%%%%%%%%%%%%%%%%%%%%%%%%%%%%%%%%%%%%%%%%%%%%%%%%%%%%%%%%%%%%%%%%%%%%%%

\newcommand{\doi}[1]{DOI: \href{http://dx.doi.org/#1}{\nolinkurl{#1}}}
\bibliographystyle{plainnat}
\bibliography{refs}

\end{document}
